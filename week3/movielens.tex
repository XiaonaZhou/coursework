% Options for packages loaded elsewhere
\PassOptionsToPackage{unicode}{hyperref}
\PassOptionsToPackage{hyphens}{url}
%
\documentclass[
]{article}
\usepackage{lmodern}
\usepackage{amssymb,amsmath}
\usepackage{ifxetex,ifluatex}
\ifnum 0\ifxetex 1\fi\ifluatex 1\fi=0 % if pdftex
  \usepackage[T1]{fontenc}
  \usepackage[utf8]{inputenc}
  \usepackage{textcomp} % provide euro and other symbols
\else % if luatex or xetex
  \usepackage{unicode-math}
  \defaultfontfeatures{Scale=MatchLowercase}
  \defaultfontfeatures[\rmfamily]{Ligatures=TeX,Scale=1}
\fi
% Use upquote if available, for straight quotes in verbatim environments
\IfFileExists{upquote.sty}{\usepackage{upquote}}{}
\IfFileExists{microtype.sty}{% use microtype if available
  \usepackage[]{microtype}
  \UseMicrotypeSet[protrusion]{basicmath} % disable protrusion for tt fonts
}{}
\makeatletter
\@ifundefined{KOMAClassName}{% if non-KOMA class
  \IfFileExists{parskip.sty}{%
    \usepackage{parskip}
  }{% else
    \setlength{\parindent}{0pt}
    \setlength{\parskip}{6pt plus 2pt minus 1pt}}
}{% if KOMA class
  \KOMAoptions{parskip=half}}
\makeatother
\usepackage{xcolor}
\IfFileExists{xurl.sty}{\usepackage{xurl}}{} % add URL line breaks if available
\IfFileExists{bookmark.sty}{\usepackage{bookmark}}{\usepackage{hyperref}}
\hypersetup{
  pdftitle={Movielens},
  hidelinks,
  pdfcreator={LaTeX via pandoc}}
\urlstyle{same} % disable monospaced font for URLs
\usepackage[margin=1in]{geometry}
\usepackage{color}
\usepackage{fancyvrb}
\newcommand{\VerbBar}{|}
\newcommand{\VERB}{\Verb[commandchars=\\\{\}]}
\DefineVerbatimEnvironment{Highlighting}{Verbatim}{commandchars=\\\{\}}
% Add ',fontsize=\small' for more characters per line
\usepackage{framed}
\definecolor{shadecolor}{RGB}{248,248,248}
\newenvironment{Shaded}{\begin{snugshade}}{\end{snugshade}}
\newcommand{\AlertTok}[1]{\textcolor[rgb]{0.94,0.16,0.16}{#1}}
\newcommand{\AnnotationTok}[1]{\textcolor[rgb]{0.56,0.35,0.01}{\textbf{\textit{#1}}}}
\newcommand{\AttributeTok}[1]{\textcolor[rgb]{0.77,0.63,0.00}{#1}}
\newcommand{\BaseNTok}[1]{\textcolor[rgb]{0.00,0.00,0.81}{#1}}
\newcommand{\BuiltInTok}[1]{#1}
\newcommand{\CharTok}[1]{\textcolor[rgb]{0.31,0.60,0.02}{#1}}
\newcommand{\CommentTok}[1]{\textcolor[rgb]{0.56,0.35,0.01}{\textit{#1}}}
\newcommand{\CommentVarTok}[1]{\textcolor[rgb]{0.56,0.35,0.01}{\textbf{\textit{#1}}}}
\newcommand{\ConstantTok}[1]{\textcolor[rgb]{0.00,0.00,0.00}{#1}}
\newcommand{\ControlFlowTok}[1]{\textcolor[rgb]{0.13,0.29,0.53}{\textbf{#1}}}
\newcommand{\DataTypeTok}[1]{\textcolor[rgb]{0.13,0.29,0.53}{#1}}
\newcommand{\DecValTok}[1]{\textcolor[rgb]{0.00,0.00,0.81}{#1}}
\newcommand{\DocumentationTok}[1]{\textcolor[rgb]{0.56,0.35,0.01}{\textbf{\textit{#1}}}}
\newcommand{\ErrorTok}[1]{\textcolor[rgb]{0.64,0.00,0.00}{\textbf{#1}}}
\newcommand{\ExtensionTok}[1]{#1}
\newcommand{\FloatTok}[1]{\textcolor[rgb]{0.00,0.00,0.81}{#1}}
\newcommand{\FunctionTok}[1]{\textcolor[rgb]{0.00,0.00,0.00}{#1}}
\newcommand{\ImportTok}[1]{#1}
\newcommand{\InformationTok}[1]{\textcolor[rgb]{0.56,0.35,0.01}{\textbf{\textit{#1}}}}
\newcommand{\KeywordTok}[1]{\textcolor[rgb]{0.13,0.29,0.53}{\textbf{#1}}}
\newcommand{\NormalTok}[1]{#1}
\newcommand{\OperatorTok}[1]{\textcolor[rgb]{0.81,0.36,0.00}{\textbf{#1}}}
\newcommand{\OtherTok}[1]{\textcolor[rgb]{0.56,0.35,0.01}{#1}}
\newcommand{\PreprocessorTok}[1]{\textcolor[rgb]{0.56,0.35,0.01}{\textit{#1}}}
\newcommand{\RegionMarkerTok}[1]{#1}
\newcommand{\SpecialCharTok}[1]{\textcolor[rgb]{0.00,0.00,0.00}{#1}}
\newcommand{\SpecialStringTok}[1]{\textcolor[rgb]{0.31,0.60,0.02}{#1}}
\newcommand{\StringTok}[1]{\textcolor[rgb]{0.31,0.60,0.02}{#1}}
\newcommand{\VariableTok}[1]{\textcolor[rgb]{0.00,0.00,0.00}{#1}}
\newcommand{\VerbatimStringTok}[1]{\textcolor[rgb]{0.31,0.60,0.02}{#1}}
\newcommand{\WarningTok}[1]{\textcolor[rgb]{0.56,0.35,0.01}{\textbf{\textit{#1}}}}
\usepackage{longtable,booktabs}
% Correct order of tables after \paragraph or \subparagraph
\usepackage{etoolbox}
\makeatletter
\patchcmd\longtable{\par}{\if@noskipsec\mbox{}\fi\par}{}{}
\makeatother
% Allow footnotes in longtable head/foot
\IfFileExists{footnotehyper.sty}{\usepackage{footnotehyper}}{\usepackage{footnote}}
\makesavenoteenv{longtable}
\usepackage{graphicx,grffile}
\makeatletter
\def\maxwidth{\ifdim\Gin@nat@width>\linewidth\linewidth\else\Gin@nat@width\fi}
\def\maxheight{\ifdim\Gin@nat@height>\textheight\textheight\else\Gin@nat@height\fi}
\makeatother
% Scale images if necessary, so that they will not overflow the page
% margins by default, and it is still possible to overwrite the defaults
% using explicit options in \includegraphics[width, height, ...]{}
\setkeys{Gin}{width=\maxwidth,height=\maxheight,keepaspectratio}
% Set default figure placement to htbp
\makeatletter
\def\fps@figure{htbp}
\makeatother
\setlength{\emergencystretch}{3em} % prevent overfull lines
\providecommand{\tightlist}{%
  \setlength{\itemsep}{0pt}\setlength{\parskip}{0pt}}
\setcounter{secnumdepth}{-\maxdimen} % remove section numbering

\title{Movielens}
\author{}
\date{\vspace{-2.5em}2020-06-15 15:44:09}

\begin{document}
\maketitle

{
\setcounter{tocdepth}{3}
\tableofcontents
}
\hypertarget{load-and-preview-data}{%
\section{Load and preview data}\label{load-and-preview-data}}

Read data from the \texttt{ratings.csv} file

\begin{Shaded}
\begin{Highlighting}[]
\NormalTok{ratings <-}\StringTok{ }\KeywordTok{read_csv}\NormalTok{(}\StringTok{'ratings.csv'}\NormalTok{,}
                    \DataTypeTok{col_names =} \KeywordTok{c}\NormalTok{(}\StringTok{'user_id'}\NormalTok{,}\StringTok{'movie_id'}\NormalTok{,}\StringTok{'rating'}\NormalTok{,}\StringTok{'timestamp'}\NormalTok{))}
\end{Highlighting}
\end{Shaded}

\begin{verbatim}
## Parsed with column specification:
## cols(
##   user_id = col_double(),
##   movie_id = col_double(),
##   rating = col_double(),
##   timestamp = col_double()
## )
\end{verbatim}

Loaded 305.2 Mb of ratings data, containing 10,000,054 ratings. Here's a
preview:

\begin{Shaded}
\begin{Highlighting}[]
\KeywordTok{head}\NormalTok{(ratings) }\OperatorTok\StringTok{ }\KeywordTok{kable}\NormalTok{()}
\end{Highlighting}
\end{Shaded}

\begin{longtable}[]{@{}rrrr@{}}
\toprule
user\_id & movie\_id & rating & timestamp\tabularnewline
\midrule
\endhead
1 & 122 & 5 & 838985046\tabularnewline
1 & 185 & 5 & 838983525\tabularnewline
1 & 231 & 5 & 838983392\tabularnewline
1 & 292 & 5 & 838983421\tabularnewline
1 & 316 & 5 & 838983392\tabularnewline
1 & 329 & 5 & 838983392\tabularnewline
\bottomrule
\end{longtable}

\hypertarget{summary-statistics}{%
\section{Summary statistics}\label{summary-statistics}}

\begin{Shaded}
\begin{Highlighting}[]
\CommentTok{# plot the distribution of rating values https://speakerdeck.com/jhofman/modeling-social-data-lecture-2-introduction-to-counting?slide=26}

\KeywordTok{ggplot}\NormalTok{(ratings,}\KeywordTok{aes}\NormalTok{(}\DataTypeTok{x=}\NormalTok{rating))}\OperatorTok{+}
\StringTok{  }\KeywordTok{geom_bar}\NormalTok{()}\OperatorTok{+}
\StringTok{  }\KeywordTok{scale_y_continuous}\NormalTok{(}\DataTypeTok{label =}\NormalTok{ comma)}\OperatorTok{+}
\StringTok{  }\KeywordTok{xlab}\NormalTok{(}\StringTok{'Rating'}\NormalTok{) }\OperatorTok{+}
\StringTok{  }\KeywordTok{ylab}\NormalTok{(}\StringTok{'Number of ratings'}\NormalTok{)}
\end{Highlighting}
\end{Shaded}

\includegraphics{movielens_files/figure-latex/dist-ratings-1.pdf}

\hypertarget{per-movie-stats}{%
\subsection{Per-movie stats}\label{per-movie-stats}}

\begin{Shaded}
\begin{Highlighting}[]
\CommentTok{# aggregate ratings by movie, computing mean and number of ratings}
\CommentTok{# hint: use the n() function for easy counting within a group}

\NormalTok{rating_mean_num <-}\StringTok{ }\NormalTok{ratings }\OperatorTok\StringTok{ }
\StringTok{  }\KeywordTok{group_by}\NormalTok{(movie_id) }\OperatorTok\StringTok{ }
\StringTok{  }\KeywordTok{summarize}\NormalTok{(}\DataTypeTok{num_rating =} \KeywordTok{n}\NormalTok{(),}
            \DataTypeTok{mean_rating =} \KeywordTok{mean}\NormalTok{(rating))}
\NormalTok{rating_mean_num}
\end{Highlighting}
\end{Shaded}

\begin{verbatim}
## # A tibble: 10,677 x 3
##    movie_id num_rating mean_rating
##       <dbl>      <int>       <dbl>
##  1        1      26449        3.93
##  2        2      12032        3.21
##  3        3       7790        3.15
##  4        4       1764        2.86
##  5        5       7135        3.08
##  6        6      13696        3.81
##  7        7       8064        3.37
##  8        8        899        3.13
##  9        9       2518        3.00
## 10       10      16918        3.43
## # ... with 10,667 more rows
\end{verbatim}

\begin{Shaded}
\begin{Highlighting}[]
\CommentTok{# plot distribution of movie popularity (= number of ratings the movie received)}
\CommentTok{# hint: try scale_x_log10() for a logarithmic x axis}

\NormalTok{rating_mean_num }\OperatorTok\StringTok{ }
\StringTok{  }\KeywordTok{ggplot}\NormalTok{(}\KeywordTok{aes}\NormalTok{(}\DataTypeTok{x =}\NormalTok{ num_rating))}\OperatorTok{+}
\StringTok{  }\KeywordTok{geom_histogram}\NormalTok{(}\DataTypeTok{bins =} \DecValTok{60}\NormalTok{)}\OperatorTok{+}
\StringTok{  }\KeywordTok{scale_x_log10}\NormalTok{()}
\end{Highlighting}
\end{Shaded}

\includegraphics{movielens_files/figure-latex/dist-movie-popularity-1.pdf}

\begin{Shaded}
\begin{Highlighting}[]
\CommentTok{# plot distribution of mean ratings by movie https://speakerdeck.com/jhofman/modeling-social-data-lecture-2-introduction-to-counting?slide=28}
\CommentTok{# hint: try geom_histogram and geom_density}
\NormalTok{rating_mean_num }\OperatorTok\StringTok{ }
\StringTok{  }\KeywordTok{ggplot}\NormalTok{(}\KeywordTok{aes}\NormalTok{(}\DataTypeTok{x =}\NormalTok{ mean_rating)) }\OperatorTok{+}
\StringTok{  }\CommentTok{#geom_histogram() +}
\StringTok{  }\KeywordTok{geom_density}\NormalTok{(}\DataTypeTok{fill=}\StringTok{"black"}\NormalTok{)}\OperatorTok{+}
\StringTok{  }\KeywordTok{xlab}\NormalTok{(}\StringTok{'Mean Rating by Movie'}\NormalTok{) }\OperatorTok{+}
\StringTok{  }\KeywordTok{ylab}\NormalTok{(}\StringTok{'Density'}\NormalTok{)}
\end{Highlighting}
\end{Shaded}

\includegraphics{movielens_files/figure-latex/dist-mean-ratings-by-movie-1.pdf}

\begin{Shaded}
\begin{Highlighting}[]
\CommentTok{# rank movies by popularity and compute the cdf, or fraction of movies covered by the top-k moves https://speakerdeck.com/jhofman/modeling-social-data-lecture-2-introduction-to-counting?slide=30}
\CommentTok{# hint: use dplyr's rank and arrange functions, and the base R sum and cumsum functions}
\CommentTok{# store the result in a new data frame so you can use it in creating figure 2 from the paper below}

\CommentTok{# plot the CDF of movie popularity}

\NormalTok{cdf_ratings <-}\StringTok{ }\NormalTok{ratings }\OperatorTok
\StringTok{  }\KeywordTok{group_by}\NormalTok{(movie_id) }\OperatorTok\StringTok{ }
\StringTok{  }\KeywordTok{summarise}\NormalTok{(}\DataTypeTok{num_rating =} \KeywordTok{n}\NormalTok{()) }\OperatorTok\StringTok{ }
\StringTok{  }\KeywordTok{arrange}\NormalTok{(}\KeywordTok{desc}\NormalTok{(num_rating)) }\OperatorTok\StringTok{ }
\StringTok{  }\KeywordTok{mutate}\NormalTok{( }\DataTypeTok{rank =} \KeywordTok{row_number}\NormalTok{()) }\OperatorTok\StringTok{ }
\StringTok{  }\KeywordTok{ungroup}\NormalTok{() }\OperatorTok\StringTok{ }
\StringTok{  }\KeywordTok{mutate}\NormalTok{(}\DataTypeTok{cdf =} \KeywordTok{cumsum}\NormalTok{(num_rating)}\OperatorTok{/}\KeywordTok{sum}\NormalTok{(num_rating))}

\KeywordTok{ggplot}\NormalTok{(cdf_ratings, }\KeywordTok{aes}\NormalTok{(}\DataTypeTok{x=}\NormalTok{rank, }\DataTypeTok{y =}\NormalTok{ cdf))}\OperatorTok{+}
\StringTok{  }\KeywordTok{geom_line}\NormalTok{() }\OperatorTok{+}
\StringTok{  }\KeywordTok{scale_y_continuous}\NormalTok{(}\DataTypeTok{label =}\NormalTok{ percent)}\OperatorTok{+}
\StringTok{  }\KeywordTok{scale_x_continuous}\NormalTok{(}\DataTypeTok{label =}\NormalTok{ comma) }\OperatorTok{+}
\StringTok{  }\KeywordTok{xlab}\NormalTok{(}\StringTok{'Movie rank'}\NormalTok{) }\OperatorTok{+}
\StringTok{  }\KeywordTok{ylab}\NormalTok{(}\StringTok{'Cumulative popularity'}\NormalTok{)}
\end{Highlighting}
\end{Shaded}

\includegraphics{movielens_files/figure-latex/cdf-movie-pop-1.pdf}

\hypertarget{per-user-stats}{%
\section{Per-user stats}\label{per-user-stats}}

\begin{Shaded}
\begin{Highlighting}[]
\CommentTok{# aggregate ratings by user, computing mean and number of ratings}

\NormalTok{user_rating_mean_num <-}\StringTok{ }\NormalTok{ratings }\OperatorTok\StringTok{ }
\StringTok{  }\KeywordTok{group_by}\NormalTok{(user_id) }\OperatorTok\StringTok{ }
\StringTok{  }\KeywordTok{summarize}\NormalTok{(}\DataTypeTok{num_rating =} \KeywordTok{n}\NormalTok{(),}
            \DataTypeTok{mean_rating =} \KeywordTok{mean}\NormalTok{(rating))}
\NormalTok{user_rating_mean_num}
\end{Highlighting}
\end{Shaded}

\begin{verbatim}
## # A tibble: 69,878 x 3
##    user_id num_rating mean_rating
##      <dbl>      <int>       <dbl>
##  1       1         22        5   
##  2       2         20        3.2 
##  3       3         33        3.94
##  4       4         38        4.03
##  5       5         87        3.85
##  6       6         42        3.93
##  7       7        109        3.93
##  8       8        800        3.40
##  9       9         24        4   
## 10      10        123        3.81
## # ... with 69,868 more rows
\end{verbatim}

\begin{Shaded}
\begin{Highlighting}[]
\KeywordTok{min}\NormalTok{(user_rating_mean_num}\OperatorTok{$}\NormalTok{num_rating)}
\end{Highlighting}
\end{Shaded}

\begin{verbatim}
## [1] 20
\end{verbatim}

\begin{Shaded}
\begin{Highlighting}[]
\CommentTok{# plot distribution of user activity (= number of ratings the user made)}
\CommentTok{# hint: try a log scale here}
\NormalTok{user_rating_mean_num }\OperatorTok\StringTok{ }
\StringTok{  }\KeywordTok{ggplot}\NormalTok{(}\KeywordTok{aes}\NormalTok{(}\DataTypeTok{x =}\NormalTok{ num_rating))}\OperatorTok{+}
\StringTok{  }\KeywordTok{geom_histogram}\NormalTok{(}\DataTypeTok{bins =} \DecValTok{60}\NormalTok{)}\OperatorTok{+}
\StringTok{  }\KeywordTok{scale_x_log10}\NormalTok{()}
\end{Highlighting}
\end{Shaded}

\includegraphics{movielens_files/figure-latex/dist-user-activity-1.pdf}

\hypertarget{anatomy-of-the-long-tail}{%
\section{Anatomy of the long tail}\label{anatomy-of-the-long-tail}}

\begin{Shaded}
\begin{Highlighting}[]
\CommentTok{# generate the equivalent of figure 2 of this paper:}
\CommentTok{# https://5harad.com/papers/long_tail.pdf}

\CommentTok{# Specifically, for the subset of users who rated at least 10 movies,}
\CommentTok{# produce a plot that shows the fraction of users satisfied (vertical}
\CommentTok{# axis) as a function of inventory size (horizontal axis). We will}
\CommentTok{# define "satisfied" as follows: an individual user is satisfied p% of}
\CommentTok{# the time at inventory of size k if at least p% of the movies they}
\CommentTok{# rated are contained in the top k most popular movies. As in the}
\CommentTok{# paper, produce one curve for the 100% user satisfaction level and}
\CommentTok{# another for 90%---do not, however, bother implementing the null}
\CommentTok{# model (shown in the dashed lines).}

\NormalTok{movie_rank <-}\StringTok{ }\NormalTok{ratings }\OperatorTok
\StringTok{  }\KeywordTok{group_by}\NormalTok{(movie_id) }\OperatorTok\StringTok{ }
\StringTok{  }\KeywordTok{summarise}\NormalTok{(}\DataTypeTok{num_rating =} \KeywordTok{n}\NormalTok{()) }\OperatorTok\StringTok{ }
\StringTok{  }\KeywordTok{arrange}\NormalTok{(}\KeywordTok{desc}\NormalTok{(num_rating)) }\OperatorTok\StringTok{ }
\StringTok{  }\KeywordTok{mutate}\NormalTok{( }\DataTypeTok{rank =} \KeywordTok{row_number}\NormalTok{()) }\OperatorTok\StringTok{ }
\StringTok{  }\KeywordTok{select}\NormalTok{(movie_id, rank)}

\NormalTok{users <-}\StringTok{ }\NormalTok{ratings }\OperatorTok\StringTok{ }
\StringTok{  }\KeywordTok{select}\NormalTok{(user_id,movie_id)}

\NormalTok{user_with_movie_rank<-}\KeywordTok{left_join}\NormalTok{(movie_rank,users, }\DataTypeTok{by=}\StringTok{"movie_id"}\NormalTok{)}

\NormalTok{user_max_rank <-}\StringTok{ }\NormalTok{user_with_movie_rank }\OperatorTok\StringTok{ }
\StringTok{  }\KeywordTok{group_by}\NormalTok{(user_id) }\OperatorTok\StringTok{ }
\StringTok{  }\KeywordTok{summarise}\NormalTok{(}\DataTypeTok{max_rank =} \KeywordTok{max}\NormalTok{(rank)) }\OperatorTok\StringTok{ }
\StringTok{  }\KeywordTok{arrange}\NormalTok{(max_rank) }\OperatorTok\StringTok{ }
\StringTok{  }\KeywordTok{select}\NormalTok{(user_id, max_rank) }\OperatorTok\StringTok{ }
\StringTok{  }\KeywordTok{group_by}\NormalTok{(max_rank) }\OperatorTok\StringTok{ }
\StringTok{  }\KeywordTok{summarise}\NormalTok{(}\DataTypeTok{count =} \KeywordTok{n}\NormalTok{()) }\OperatorTok\StringTok{ }
\StringTok{  }\KeywordTok{mutate}\NormalTok{(}\DataTypeTok{cdf =} \KeywordTok{cumsum}\NormalTok{(count)}\OperatorTok{/}\KeywordTok{sum}\NormalTok{(count))}
\KeywordTok{ggplot}\NormalTok{(user_max_rank, }\KeywordTok{aes}\NormalTok{(}\DataTypeTok{x=}\NormalTok{max_rank, }\DataTypeTok{y =}\NormalTok{ cdf))}\OperatorTok{+}
\StringTok{  }\KeywordTok{geom_line}\NormalTok{() }\OperatorTok{+}
\StringTok{  }\KeywordTok{scale_y_continuous}\NormalTok{(}\DataTypeTok{label =}\NormalTok{ percent)}\OperatorTok{+}
\StringTok{  }\KeywordTok{scale_x_continuous}\NormalTok{(}\DataTypeTok{label =}\NormalTok{ comma) }\OperatorTok{+}
\StringTok{  }\KeywordTok{xlab}\NormalTok{(}\StringTok{'Inventory Size'}\NormalTok{) }\OperatorTok{+}
\StringTok{  }\KeywordTok{ylab}\NormalTok{(}\StringTok{'Percent of Users Satisfied(100%)'}\NormalTok{)}
\end{Highlighting}
\end{Shaded}

\includegraphics{movielens_files/figure-latex/long-tail-1.pdf}

\begin{Shaded}
\begin{Highlighting}[]
\NormalTok{user_}\DecValTok{90}\NormalTok{_rank <-}\StringTok{ }\NormalTok{user_with_movie_rank }\OperatorTok\StringTok{ }
\StringTok{  }\KeywordTok{group_by}\NormalTok{(user_id) }\OperatorTok\StringTok{ }
\StringTok{  }\KeywordTok{summarise}\NormalTok{(}\DataTypeTok{rank_90 =} \KeywordTok{quantile}\NormalTok{(rank,}\FloatTok{0.9}\NormalTok{)) }\OperatorTok
\StringTok{  }\KeywordTok{arrange}\NormalTok{(rank_}\DecValTok{90}\NormalTok{) }\OperatorTok
\StringTok{  }\KeywordTok{select}\NormalTok{(user_id, rank_}\DecValTok{90}\NormalTok{) }\OperatorTok
\StringTok{  }\KeywordTok{group_by}\NormalTok{(rank_}\DecValTok{90}\NormalTok{) }\OperatorTok
\StringTok{  }\KeywordTok{summarise}\NormalTok{(}\DataTypeTok{count =} \KeywordTok{n}\NormalTok{()) }\OperatorTok
\StringTok{  }\KeywordTok{mutate}\NormalTok{(}\DataTypeTok{cdf =} \KeywordTok{cumsum}\NormalTok{(count)}\OperatorTok{/}\KeywordTok{sum}\NormalTok{(count))}


\KeywordTok{ggplot}\NormalTok{(user_}\DecValTok{90}\NormalTok{_rank, }\KeywordTok{aes}\NormalTok{(}\DataTypeTok{x=}\NormalTok{rank_}\DecValTok{90}\NormalTok{, }\DataTypeTok{y =}\NormalTok{ cdf))}\OperatorTok{+}
\StringTok{  }\KeywordTok{geom_line}\NormalTok{() }\OperatorTok{+}
\StringTok{  }\KeywordTok{scale_y_continuous}\NormalTok{(}\DataTypeTok{label =}\NormalTok{ percent)}\OperatorTok{+}
\StringTok{  }\KeywordTok{scale_x_continuous}\NormalTok{(}\DataTypeTok{label =}\NormalTok{ comma) }\OperatorTok{+}
\StringTok{  }\KeywordTok{xlab}\NormalTok{(}\StringTok{'Inventory Size'}\NormalTok{) }\OperatorTok{+}
\StringTok{  }\KeywordTok{ylab}\NormalTok{(}\StringTok{'Percent of Users Satisfied(90%)'}\NormalTok{)}
\end{Highlighting}
\end{Shaded}

\includegraphics{movielens_files/figure-latex/long-tail-2.pdf}

\end{document}
